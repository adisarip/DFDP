\section{Related Work}
There are two lines of work in literature which are broadly related to the problem we have outlined. The first line of work looks into various aspects of data freshness as a data quality metric in systems such as Web Servers~\cite{labrinidis2004exploring}, Real-Time Databases~\cite{wei2004maintaining,kang2004managing}, etc. A second aspect which is important in our work is the selection of parameters of a set of real-time tasks such that the tasks are schedulable and particular system objectives are optimized. However, there are few works which have combined the problem of selecting periods of real-time tasks while guaranteeing end-to-end data freshness. We now present the differences between existing research and the work which we present.

In~\cite{bouzeghoub2004framework,peralta2006data}, the authors present a comprehensive overview of freshness as a data quality metric and also a framework for analysis of freshness. Freshness is defined with respect to two notions, namely \emph{currency factor} and \emph{timeliness factor}. The currency factor represents how stale a data is with respect to its source, while the timeliness factor represents how old a data is since its creation/update at the source. Several metrics are described to measure freshness according to these definitions. There are also works which have considered freshness related problems in real-time databases. In~\cite{wei2004maintaining}, the effects of update policies to maintain data freshness of derived data was studied in the context of a distributed real-time database and a novel update policy was proposed. Conversely, an immediate update policy based on a QoS management architecture is used by~\cite{kang2004managing}. However, none of the above works consider freshness in a task context or deriving parameters to maintain the target freshness constraint.

The problem addressed in~\cite{bedewy2016optimizing} was to optimize data freshness along with other objectives such as throughput in a multi-server information-update system. The authors propose a preemptive Last-Come First-Served policy and show that it optimizes freshness, throughput, and delay performance in infinite buffer systems. A deferrable scheduling algorithm is proposed in~\cite{xiong2005deferrable} for maintaining data freshness so as to minimize the update workload. The sampling time of a transaction job is deferred as late as possible while guaranteeing the temporal validity of the data. In contrast, our work provides an analytical framework to derive the period of tasks in order to provide the required end-to-end data freshness constraint and does not restrict the scheduling policies.

While this work may appear similar to precedence-constrained scheduling and task dependency problems there are key differences. We do not assign any task precedence and intentionally allow jobs to execute at any time and in any order for scheduler generality. A freshness guarantee may hold without precedence between producer and consumer tasks. We also make no assumptions about scheduler attributes such as preemptability. Precedence-constrained scheduling works typically consider best-effort, batch task sets instead of real-time, periodic task sets. Regardless, task dependencies and precedence for a coarse ordering of tasks and provide no timing guarantees while we aim to bound the time between tasks in a chain and allow for repeated tasks in the chain. By doing so we allow for greater abstraction of the scheduler, although task dependencies or precedence could be used in conjunction with our system to prevent, for example, a consumer running directly before a producer. In this case the correctness of the formulation holds, although accounting for the precedence in the schedulability test would provide better bounds and more accurate schedulability decisions. Finally, this work provides a correctness by construction technique and does not attempt to determine a schedule for task sets with known parameters as do most precedence-constrained scheduling problems.

There are prior works that look at choosing task periods to meet individual latency requirements in real-time systems using dynamic priority~\cite{seto1996task} and static priority~\cite{SetoLehoczkySha} scheduling methods. In~\cite{ChantemWangLemmonHu}, the authors propose a heuristic to derive a feasible period-deadline combination such that the task set becomes schedulable under the assumption that task deadlines are piecewise first-order differentiable functions of the respective periods. Wu et. al.~\cite{wu2010parameter} presented an approach to select the task periods and deadlines, under EDF scheduling, to enhance the control performance of a system. However, the above works do not consider data freshness as a constraint in their frameworks.

There is one work~\cite{gerber1995guaranteeing} that has looked at the problem of period selection combined with data freshness requirement. Three classes of timing constraints are considered in~\cite{gerber1995guaranteeing} namely freshness, correlation, and separation. An iterative pruning-based heuristic is used to derive the period, offset, and deadline of tasks such that the end-to-end constraints are met. In contrast, our work proposes an analytical framework to derive the task periods remaining completely agnostic to the scheduling strategy. Moreover, in~\cite{gerber1995guaranteeing}, the periods considered for producer tasks were harmonic with respect to the periods of the consumer tasks. We do not impose such a restriction, which may allow for scheduling more task sets and allows more flexibility when minimizing our optimization objective.

While our notion of freshness and task chains may evoke thoughts of task dependencies and precendence-constrained scheduling



