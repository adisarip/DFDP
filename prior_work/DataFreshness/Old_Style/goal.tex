\section{Goal}

In broad terms, the goal we wish to achieve is for the input of a given task to meet predetermined freshness guarantees, where freshness represents the age of the data. An example of this would be a task $B$ which needs to use speed sensor data produced by a task A that is at most 100 milliseconds old.

The notion of freshness, or alternatively staleness, is not a new concept. A wide variety of domains consider this notion, with the definition tailored to the domain, all of which agree that in their domains the quality of data is not merely a function of its "correctness" or accuracy. Common notions use wording such as currency \cite{Segev1990}, which describes how much time has passed since data collection or generation, and timeliness \cite{Wang:1996:BAD:1189570.1189572}, which measures how old the data is when collected. For our purposes, these two are essentially the same notion. We will consider freshness to be the time elapsed since the data was produced, particularly when it is output by a generation or collection task.

In particular, we are going to examine the following scenario: given a task $Z$ with fixed period that consumes data that makes its way through a task chain $A, B, C, \ldots$, choose the periods for $A, B, C, \ldots$ such as to ensure our freshness bound is always enforced. That is to say, when task $Z$ runs the age of the data created by task $A$ is less than our freshness bound.

The solution to this problem is not always unique. Given a set of possible solutions, we wish to rate them against some metric to select one that best fits our needs. There are several metrics one could use; in this work we focus on minimizing total task set utilization. More precisely, we will attempt to minimize maximum system utilization. We chose this metric because it is a common metric of schedulability and efficiency in the real-time systems sector. Intuitively, a low task set utilization provides the necessary performance at the lowest computational cost, allowing for more tasks to be introduced to the system and increasing the likelihood of schedulability.